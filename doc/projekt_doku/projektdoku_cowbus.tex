\documentclass{IEEEtran}
\usepackage[utf8]{inputenc}
\usepackage{acronym}
\usepackage{footnote}
\usepackage{algorithmic}
\usepackage[autostyle=true,german=quotes]{csquotes}


\long\def\comment / *#1* /{}


\begin{document}


\title{cowbus -- a small home automation bus}
\author{Robin~Backhaus \and Patrick~Kanzler \and Josef~Schnurrer \and Michael~Zapf}
\date{\today}



\maketitle

\begin{abstract}
\end{abstract}


\section{Einleitung}

\enquote{Smart Home} ist ein Schlagwort, um das man im 21. Jahrhundert
kaum noch herum kommt. Der moderne Mensch möchte sein Zuhause vernetzen.
Lichtschalter sind nicht mehr nur noch passive Komponenten in einer
Elektroinstallation, sie sind aktive Kommunikationspartner in einem großen Netz.

Das Licht wird nicht nur am Schalter, sondern bequem von der Couch aus mit dem
Smartphone geschalten, eingehende E-Mails erscheinen nebenbei im Fernsehbild,
die Verdunklung fährt automatisch herunter sobald die Sonne blendet --
all das ist keine Science Fiction mehr sondern längst Realität.

\section{Motivaton}

TODO: Warum noch ein Hausbus? Vielleicht?

\section{Projektziele und Ideen}

Mesh-Netzwerk
dezentrale Organisation
Beispiel-Sensor
Beispiel-Aktor

    \subsection{Optionale Ziele}

6LoWPAN
Ethernet-Gateway
OpenHAB

\section{Projektphasen und Meilensteine}
    \subsection{Planung}
    \subsection{Konkretisierung der Ziele}
    \subsection{Kennenlernen von RIOT OS}

\url{http://www.riot-os.org/} \enquote{The friendly Operating System for the Internet of Things.}, Multithreading, gut portierbar (Schichtenarchitektur), Lizenz: LGPLv2

    \subsection{Funkchip nrf24l01+}

2,4 GHz (ISM-Band), bis zu 2 Mbps, handliches Modul

    \subsection{Erstes Prototyping mit Discovery Boards}
    \subsection{Erste Funkübertragung und Messung der Funkreichweite}
    \subsection{ARM Cortex M0: STM32F030C8T6}

bis zu 48 MHz, Gehäuse LQFP48 (7x7 mm)

    \subsection{Platinendesign}

RGB-LED, Batterie-/Akkubetrieb oder USB, drei Taster, Board ca 4 cm Radius

    \subsection{Zwischenpresentation}
    

\section{Verwendete Programmiersprachen und Techniken}

\section{Verwendete Software und Hilfsmittel}
    \subsection{KiCad}

\section{Aufgetretene Hindernisse und Risiken}

\section{Systemaufbau}
    \subsection{Funkmedium}
        (Broadcast-Medium)

    \subsection{Komponenten}
        \subsubsection{Knoten (Aktor/Sensor)}
        \subsubsection{Gateway (im Moment der PI)}
    
    \subsection{Paketaufbau}

\section{Knoten-Prototyp-Hardware}
*Wahl des Prozessors

*Wahl fiel auf STM, weil 8bit zu wenig Dampf hat, ARM hat solide Toolchain

*weiterer Vorteil: Discovery-Boards günstig, um sofort mit der Entwicklung anzufangen



*parallel dazu Entwicklung richtiger Hardware

*Anforderungen:

*Idee: Schaltermodul, welches in eine Unterputzdose passt --> bedingt Form

*möglichst günstig

*...



*

    Platine, Aufbau, Designentscheidungen, ...


\section{Software und System}
    \subsection{Kommunikationsstruktur: Nachrichtenbasiert, jeder Knoten entscheidet anhand Adresse ob er interessiert ist am Inhalt}
    \subsection{RIOT - Was, Warum; evtl. auch kritische Auseinandersetzung?}

\section{Ausblick}
Was geplant ist: CSMA/CA, ...;
kritische Auseinandersetzung mit getätigten Design-Entscheidungen

\section{Zusammenfassung}



\section*{Abkürzungen}
\renewcommand{\IEEEiedlistdecl}{\IEEEsetlabelwidth{CSMA/CA}}
\begin{acronym}
    \acro{CSMA/CA}{Carrier Sense Multiple Access with Collision Avoidance}
\end{acronym}
\renewcommand{\IEEEiedlistdecl}{\relax}% remember to reset \IEEEiedlistdecl


\comment / *
\listoffigures
\clearpage

\listoftables
\clearpage
* /

\nocite*
\bibliographystyle{IEEEtran}
\bibliography{IEEEabrv,projektdoku_cowbus}

\end{document}
