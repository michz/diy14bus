\documentclass{beamer}
\usepackage[T1]{fontenc}
\usepackage[ngerman]{babel}
\usepackage[utf8]{inputenc}
\usepackage[babel,german=quotes]{csquotes}
\usepackage{amsmath,amsfonts,amssymb}
\usepackage{hyperref}
\usepackage{lmodern}

\usetheme{JuanLesPins}
\usepackage{beamerthemeshadow}

\usepackage[sort&compress,comma,super,square]{natbib}
%\bibliographystyle{apalike} % Or your specific bibliographystyle
\bibliographystyle{unsrtdin}
%\bibliographystyle{gerplain}

\usepackage{tikz}
\usepackage[active,tightpage]{preview}
\PreviewEnvironment{tikzpicture}
\setlength\PreviewBorder{7pt}
\definecolor{lavander}{cmyk}{0,0.48,0,0}
\definecolor{violet}{cmyk}{0.79,0.88,0,0}
\definecolor{burntorange}{cmyk}{0,0.52,1,0}

\def\lav{lavander!90}
\def\oran{orange!30}

\tikzstyle{peers}=[draw,circle,violet,bottom color=\lav,
                  top color= white, text=violet,minimum width=10pt]
\tikzstyle{superpeers}=[draw,circle,burntorange, left color=\oran,
                       text=violet,minimum width=20pt]
\tikzstyle{legendsp}=[rectangle, draw, burntorange, rounded corners,
                     thin,bottom color=\oran, top color=white,
                     text=burntorange, minimum width=2.5cm]
\tikzstyle{legendp}=[rectangle, draw, violet, rounded corners, thin,
                     bottom color=\lav, top color= white,
                     text= violet, minimum width= 2.5cm]
\tikzstyle{legend_general}=[rectangle, rounded corners, thin,
                           burntorange, fill= white, draw, text=violet,
                           minimum width=2.5cm, minimum height=0.8cm]


\title{cowbus}
\author[R. Backhaus, K. Höllring, P. Kanzler, J. Schnurrer, M. Zapf]{Robin Backhaus \and Kevin Höllring \and Patrick Kanzler \and Josef Schnurrer \and Michael Zapf}
\date{28.01.2015}


\begin{document}
\frame{\titlepage}

\section{TODO}
\begin{frame}
    \frametitle{Ziele und Ideen}

    \begin{itemize}
        \item Mesh-Netzwerk
        \item dezentrale Organisation
        \item evtl. 6LoWPAN-basiert
        \item Gateway zu (vorhandenem) Ethernet \\
            (z.B. zur Nutzung mit \emph{OpenHAB})
    \end{itemize}

    \leavevmode
    \makebox(0,0){\put(220,0){
        \includegraphics[scale=0.3]{img/openhab-logo-square.png}
    }}
\end{frame}

\begin{frame}
    \frametitle{Prinzip}

    \begin{itemize}
        \item TODO: Grafik mit Lichtschaltern/Sensoren und Lampen/Aktoren
            \begin{itemize}
                \item TODO
            \end{itemize}
        \item
        \item
    \end{itemize}
    
\end{frame}

%\begin{frame}    
%    %schamlos geklaut, vielleicht nützlich um eine schönes Bildchen zu malen
%\begin{tikzpicture}[auto, thick]
%  % Place super peers and connect them
%  \foreach \place/\name in {{(0,-1)/a}, {(2,0)/b}, {(2,2)/c}, {(0,2)/d},
%           {(-2,0)/e}}
%    \node[superpeers] (\name) at \place {};
%  \foreach \source/\dest in {a/b, a/c, a/d, b/c, c/d,a/e,d/e}
%    \path (\source) edge (\dest);
%   %
%   % Place normal peers
%  \foreach \pos/\i in {above left of/1, left of/2, below left of/3}
%    \node[peers, \pos = e] (e\i) {};
%   \foreach \speer/\peer in {e/e1,e/e2,e/e3}
%    \path (\speer) edge (\peer);
%   %
%   \foreach \pos/\i in {above right of/1, right of/2, below right of/3}
%    \node[peers, \pos =b ] (b\i) {};
%   \foreach \speer/\peer in {b/b1,b/b2,b/b3}
%   \path (\speer) edge (\peer);
%   %
%   \node[peers, above of=d] (d1){};
%   \path (d) edge (d1);
%   %
%   \foreach \pos/\i in {below left of/1, below of/2}
%   \node[peers, \pos =a ] (a\i) {};
%   \foreach \speer/\peer in {a/a1,a/a2}
%   \path (\speer) edge (\peer);
%   %%%%%%%%
%   % Legends
%   \node[legendsp] at (5,0) {\small{Super-peers}};
%   \node[legendp] at (5,2) {\small{Normal peers}};
%   \node[legend_general] at (0,4) {\small{\textsc{Skype-topology}}};
%\end{tikzpicture}
%\end{frame}

\begin{frame}
    \frametitle{Hardware}

    \begin{itemize}
        \item TODO schönes Bildchen
            \begin{itemize}
                \item TODO
            \end{itemize}
        \item Batteriebetrieben (oder USB)
        \item ARM Cortex M0
        \item RGB-LED und normale LED, 3 Taster
    \end{itemize}
\end{frame}

\begin{frame}
    \frametitle{Firmware}

    \begin{itemize}
        \item TODO: RIOT-basiert
            \begin{itemize}
                \item TODO
            \end{itemize}
        \item
        \item
    \end{itemize}
\end{frame}

\begin{frame}
    \frametitle{Gateway}

    \begin{itemize}
        \item Kommunikation mit IP-Netzen
        \item erste Variante: Raspberry Pi mit Funkmodul
            \begin{itemize}
                \item Selbst Knoten im Funknetz
                \item Übersetzt Nachrichten: IP $\Leftrightarrow$ Funk
                \item    TODO
            \end{itemize}
        \item
        \item
    \end{itemize}
\end{frame}

%\begin{block}{Blocktitel}
%Blocktext
%\end{block}
%
%\begin{exampleblock}{Blocktitel}
%Blocktext
%\end{exampleblock}
%
%
%\begin{alertblock}{Blocktitel}
%Blocktext
%\end{alertblock}
%}

\nocite*
\bibliography{2015-01-28_cowbus}{}

\end{document}

