\documentclass{IEEEtran}
\usepackage[utf8]{inputenc}
\usepackage{acronym}
\usepackage{footnote}
\usepackage{algorithmic}
\usepackage[autostyle=true,german=quotes]{csquotes}


\long\def\comment / *#1* /{}


\begin{document}


\title{cowbus -- a small home automation bus}
\author{Robin~Backhaus \and Patrick~Kanzler \and Josef~Schnurrer \and Michael~Zapf}
\date{\today}



\maketitle

\begin{abstract}
\end{abstract}


\section{Einleitung}
\enquote{Smart Home} ist ein Schlagwort, um das man im 21. Jahrhungedert
kaum noch herum kommt. Der moderne Mensch möchte sein Zuhause vernetzen.
Lichtschalter sind nicht mehr nur noch passive Komponenten in einer
Elektroinstallation, sie sind aktive Kommunikationspartner in einem großen Netz.

Das Licht wird nicht nur am Schalter, sondern bequem von der Coach aus mit dem
Smartphone geschalten, eingehende E-Mails erscheinen nebenbei im Fernsehbild,
die Verdunklung fährt automatisch herunter sobald die Sonne blendet --
all das ist keine Science Fiction mehr sondern längst Realität.

\section{Motivaton}
TODO: Warum noch ein Hausbus?


\section{Ausblick}
Was geplant ist: CSMA/CA, ...

\section{Zusammenfassung}



\section*{Abkürzungen}
\renewcommand{\IEEEiedlistdecl}{\IEEEsetlabelwidth{CSMA/CA}}
\begin{acronym}
    \acro{CSMA/CA}{Carrier Sense Multiple Access with Collision Avoidance}
\end{acronym}
\renewcommand{\IEEEiedlistdecl}{\relax}% remember to reset \IEEEiedlistdecl


\comment / *
\listoffigures
\clearpage

\listoftables
\clearpage
* /

\nocite*
\bibliographystyle{IEEEtran}
\bibliography{IEEEabrv,projektdoku_cowbus}

\end{document}
