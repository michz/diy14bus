\documentclass{beamer}
\usepackage[T1]{fontenc}
\usepackage[ngerman]{babel}
\usepackage[utf8]{inputenc}
\usepackage[babel,german=quotes]{csquotes}
\usepackage{amsmath,amsfonts,amssymb}
\usepackage{hyperref}
\usepackage{lmodern}

\usetheme{JuanLesPins}
\usepackage{beamerthemeshadow}

\usepackage[sort&compress,comma,super,square]{natbib}
%\bibliographystyle{apalike} % Or your specific bibliographystyle
\bibliographystyle{unsrtdin}
%\bibliographystyle{gerplain}


\title{Kleiner drahtloser Hausbus}
\author{\href{mailto:patrick.kanzler@e-technik.stud.uni-erlangen.de}{Patrick Kanzler}, \href{mailto:michael.zapf@fau.de}{Michael Zapf}}
\date{21.10.2014}


\begin{document}
\frame{\titlepage}

\section{Projektidee: Kleiner drahtloser Hausbus}
\begin{frame}
    \frametitle{Ziele und Ideen}

    \begin{itemize}
        \item Mesh-Netzwerk
        \item dezentrale Organisation
        \item evtl. 6LoWPAN-basiert
        \item Gateway zu (vorhandenem) Ethernet \\
            (z.B. zur Nutzung mit \emph{OpenHAB})
    \end{itemize}

    \leavevmode
    \makebox(0,0){\put(220,0){
        \includegraphics[scale=0.3]{openhab-logo-square.png}
    }}
\end{frame}

\begin{frame}
    \frametitle{Werbeblock: OpenHAB}

    \begin{block}{What is openHAB?}
        \small{
        openHAB is a software for integrating different home automation
        systems and technologies into one single solution that allows
        over-arching automation rules
        and that offers uniform user interfaces.}
        \flushright{\tiny{
            (Quelle: \url{http://www.openhab.org/features.html})
        }}
    \end{block}


    \begin{itemize}
        \item herstellerunabhängig
        \item open source
        \item Oberflächen beliebig gestaltbar \\
            (und abrufbar: PC, Tablet, iOS, Android, ...)
        \item untersützt bereits jetzt \\
            viele Komponenten und Systeme
    \end{itemize}

    \leavevmode
    \makebox(0,0){\put(240,80){
        \includegraphics[scale=0.25]{openhab-logo-square.png}
    }}

\end{frame}


\begin{frame}
    \frametitle{Ziele~-~Konkret}

    \begin{itemize}
        \item Beispiel-Sensor (z.B. Temperatur)
            \begin{itemize}
                \item Elektronik
                \item Gehäuse (gedruckt)
                \item Software
            \end{itemize}
        \item Beispiel-Aktor (z.B. LED an/aus)
            \begin{itemize}
                \item Elektronik
                \item Gehäuse (gedruckt)
                \item Software
            \end{itemize}
        \item Gateway ("'edge router"')
            \begin{itemize}
                \item Elektronik
                \item Gehäuse (Acryl gelasert)
                \item Software
            \end{itemize}
    \end{itemize}
\end{frame}

%\begin{block}{Blocktitel}
%Blocktext
%\end{block}
%
%\begin{exampleblock}{Blocktitel}
%Blocktext
%\end{exampleblock}
%
%
%\begin{alertblock}{Blocktitel}
%Blocktext
%\end{alertblock}
%}

\nocite*
\bibliography{2014-10-22_Projektidee}{}

\end{document}

